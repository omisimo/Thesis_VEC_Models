% !TEX encoding = UTF-8 Unicode


\addcontentsline{toc}{chapter}{Introduction} 



Un gasto público ineficiente y de impacto limitado no puede admitirse en un país que subsiste con tantas carencias sociales, económicas y que enfrenta un escenario de recursos públicos escasos. Es por esta razón que en el presente trabajo se investiga la existencia de una relación estable de largo plazo entre factores que denotan la capacidad productiva de la economía mexicana, tales como el Producto Interno Bruto, la Población Total y la Recaudación Impositiva, con factores que denotan el impacto de las políticas púbicas en la sociedad mexicana tales como el Desempleo, Gasto en Salud Pública y el Gasto en Educación Pública. Analizar las relaciones de largo plazo entre los factores anteriormente mencionados busca contribuir con evidencia empírica a la toma de decisiones que posteriormente se verán reflejadas en propuestas de políticas públicas que mejoren la gestión de los recursos y el bienestar de los ciudadanos. Las series se analizan en búsqueda de una relación de cointegración, que en caso de existir, se extenderá dicha relación a un modelo de corrección de error vectorial, mejor conocido como VECM por sus siglas en inglés.\bigskip

El presente trabajo se estructura como sigue: el capítulo $1$ muestran la descripción de las series anteriormente descritas que se emplearán en el análisis, su graficación histórica y una breve interpretación histórica de sus respectivas tendencias. Este análisis es relevante para resaltar la importancia de las series en cuestión y será de utilidad para los siguientes capítulos. \bigskip

Seguidamente, el capítulo $2$ presenta los conceptos básicos de una serie de tiempo, así como la teoría requerida para realizar las pruebas de Dickey-Fuller aumentada, las cuales nos ayudan a identificar la presencia de raíces unitarias en las series de tiempo, siendo así, el primer paso previo al análisis multivariado. \bigskip


El capítulo $3$ inicia con una breve introducción del concepto de cointegración, en primer lugar, bajo el escenario de cointegración entre dos series de tiempo para que después estos resultados se lleven a el caso más complejo, en el cual hay existencia de cointegración entre dos o más series. En segundo lugar, la teoría para la estimación de los estadísticos de la traza y del máximo valor propio que nos permitirán determinar el rango de cointegración, junto con el el proceso de estimación sobre los coeficientes del modelo VECM.\bigskip

Finalmente, el capítulo $4$ presenta el estudio del caso práctico de la estimación del modelo de corrección del error presentado en el capítulo 2 y 3. En primer lugar, determinando el rango de cointegración, seguido de la estimación del modelo VECM, para así terminar con una interpretación de los resultados obtenidos. El trabajo finaliza con una breve sección de conclusiones. 




