% !TEX encoding = UTF-8 Unicode

\addcontentsline{toc}{chapter}{Conclusiones} 


En secciones anteriores se analizó la existencia de una relación estable de largo plazo, así como la factibilidad de utilizar modelos de corrección de error (VECM) entre factores que denotan la capacidad productiva de la economía mexicana, tales como el Producto Interno Bruto y la Recaudación Impositiva, con otros factores que reflejan el impacto de las políticas públicas en la sociedad como puede ser el Gasto en Salud Pública y el Gasto en Educación Pública.  En esta sección, se discuten tanto las implicaciones y el alcance de los resultados obtenidos, como el trabajo futuro derivado del análisis.  \bigskip

El primer resultado de interés que tiene un impacto directo en el análisis y que se encuentra fuera de nuestro alcance, consiste en la dificultad para obtener información histórica de datos públicos en México. Información del gasto en servicios públicos como la educación y la salud, junto con la recaudación impositiva deberían ser datos con suficiente profundidad histórica, con periodicidad mayor a la anual y de fácil acceso. No obstante, encontrar información estructurada resulta muy complicado en fuentes oficiales nacionales, mientras que fuentes internacionales como el banco mundial ofrecen una mejor perspectiva. Sin duda, se trata de un área con una gran oportunidad de mejora.\bigskip

Las pruebas de estacionariedad de las series en conjunto con las pruebas de cointegración nos permiten cumplir con otro de los propósitos del presente trabajo, el cual consiste en analizar la posibilidad de realizar el ajuste de un modelo VECM con las series anteriormente descritas. Si bien una vez ajustado el modelo, los resultados mostrados por el mismo fueron acordes con lo indicado por la lógica (mayor recaudación fiscal y mayor crecimiento del PIB implican un mayor gasto en servicios públicos) el modelo conduce a otras dos conclusiones de importancia. En primer lugar, el modelo no solo nos muestra que las relaciones son tal y como se esperan sino que nos presenta la posibilidad de cuantificar y de medir dichas relaciones y efectos, la velocidad de ajuste que tendrán las series en caso de romperse el equilibrio, así como la tendencia que deberá llevar cada una de ellas para regresar al mismo. En segundo lugar, nos demuestra que a pesar de existir una relación de cointegración entre las series, ésta no resulta ser significativa para cada una de ellas, ya que existen series que tienen vida propia y no reaccionan ante la ruptura del equilibrio de la relación de cointegración. En el caso del PIB, esto puede ser causado porque la dinámica de las series a su vez son afectadas por otras series exógenas. En cuanto al gasto en salud, la serie parece no responder al desequilibrio, ya que se trata de un sector muy delicado para la sociedad que no puede estar indexado al crecimiento del país. Además,  a pesar de que la economía presente incrementos, éstos no se verán reflejados fácilmente en el gasto en salud mientras no exista una mejor distribución de los recursos eficientando los altos costos administrativos, no exista competencia en la provisión de los servicios ni competencia en la organización y administración del cuidado de la salud y no exista un modelo de remuneración de los servicios de salud que reflejen las actividades generadas.\bigskip


Finalmente, como trabajo futuro, se propone el análisis de la relación de cointegración conforme se publiquen más datos de las series, ya que por ahora la relación de cointegración se mantiene fuera de la banda establecida de $\pm 1$ desviación estándar, por lo que resulta de interés la evolución de las mismas, así como las decisiones gubernamentales que provoquen el regreso a la banda de equilibrio.\bigskip
